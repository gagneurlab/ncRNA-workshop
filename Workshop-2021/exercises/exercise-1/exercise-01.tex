\documentclass[]{article}
\usepackage{lmodern}
\usepackage{amssymb,amsmath}
\usepackage{ifxetex,ifluatex}
\usepackage{fixltx2e} % provides \textsubscript
\ifnum 0\ifxetex 1\fi\ifluatex 1\fi=0 % if pdftex
  \usepackage[T1]{fontenc}
  \usepackage[utf8]{inputenc}
\else % if luatex or xelatex
  \ifxetex
    \usepackage{mathspec}
  \else
    \usepackage{fontspec}
  \fi
  \defaultfontfeatures{Ligatures=TeX,Scale=MatchLowercase}
\fi
% use upquote if available, for straight quotes in verbatim environments
\IfFileExists{upquote.sty}{\usepackage{upquote}}{}
% use microtype if available
\IfFileExists{microtype.sty}{%
\usepackage{microtype}
\UseMicrotypeSet[protrusion]{basicmath} % disable protrusion for tt fonts
}{}


\usepackage{longtable,booktabs}
\usepackage{graphicx}
% grffile has become a legacy package: https://ctan.org/pkg/grffile
\IfFileExists{grffile.sty}{%
\usepackage{grffile}
}{}
\makeatletter
\def\maxwidth{\ifdim\Gin@nat@width>\linewidth\linewidth\else\Gin@nat@width\fi}
\def\maxheight{\ifdim\Gin@nat@height>\textheight\textheight\else\Gin@nat@height\fi}
\makeatother
% Scale images if necessary, so that they will not overflow the page
% margins by default, and it is still possible to overwrite the defaults
% using explicit options in \includegraphics[width, height, ...]{}
\setkeys{Gin}{width=\maxwidth,height=\maxheight,keepaspectratio}
\IfFileExists{parskip.sty}{%
\usepackage{parskip}
}{% else
\setlength{\parindent}{0pt}
\setlength{\parskip}{6pt plus 2pt minus 1pt}
}
\setlength{\emergencystretch}{3em}  % prevent overfull lines
\providecommand{\tightlist}{%
  \setlength{\itemsep}{0pt}\setlength{\parskip}{0pt}}
\setcounter{secnumdepth}{5}

%%% Use protect on footnotes to avoid problems with footnotes in titles
\let\rmarkdownfootnote\footnote%
\def\footnote{\protect\rmarkdownfootnote}

%%% Change title format to be more compact
\usepackage{titling}

% Create subtitle command for use in maketitle
\providecommand{\subtitle}[1]{
  \posttitle{
    \begin{center}\large#1\end{center}
    }
}

\setlength{\droptitle}{-2em}

\RequirePackage[]{/data/nasif12/modules_if12/SL7/i12g/R/4.0.1-Bioc3.11/lib64/R/library/BiocStyle/resources/tex/Bioconductor}

\bioctitle[]{Exercise sheet: Day 1}
    \pretitle{\vspace{\droptitle}\centering\huge}
  \posttitle{\par}
\author{Vangelis Theodorakis, Fatemeh Behjati, Julien Gagneur, Marcel Schulz}
    \preauthor{\centering\large\emph}
  \postauthor{\par}
      \predate{\centering\large\emph}
  \postdate{\par}
    \date{09 April, 2021}

% code highlighting
\definecolor{fgcolor}{rgb}{0.251, 0.251, 0.251}
\newcommand{\hlnum}[1]{\textcolor[rgb]{0.816,0.125,0.439}{#1}}%
\newcommand{\hlstr}[1]{\textcolor[rgb]{0.251,0.627,0.251}{#1}}%
\newcommand{\hlcom}[1]{\textcolor[rgb]{0.502,0.502,0.502}{\textit{#1}}}%
\newcommand{\hlopt}[1]{\textcolor[rgb]{0,0,0}{#1}}%
\newcommand{\hlstd}[1]{\textcolor[rgb]{0.251,0.251,0.251}{#1}}%
\newcommand{\hlkwa}[1]{\textcolor[rgb]{0.125,0.125,0.941}{#1}}%
\newcommand{\hlkwb}[1]{\textcolor[rgb]{0,0,0}{#1}}%
\newcommand{\hlkwc}[1]{\textcolor[rgb]{0.251,0.251,0.251}{#1}}%
\newcommand{\hlkwd}[1]{\textcolor[rgb]{0.878,0.439,0.125}{#1}}%
\let\hlipl\hlkwb
%
\usepackage{fancyvrb}
\newcommand{\VerbBar}{|}
\newcommand{\VERB}{\Verb[commandchars=\\\{\}]}
\DefineVerbatimEnvironment{Highlighting}{Verbatim}{commandchars=\\\{\}}
%
\newenvironment{Shaded}{\begin{myshaded}}{\end{myshaded}}
% set background for result chunks
\let\oldverbatim\verbatim
\renewenvironment{verbatim}{\color{codecolor}\begin{myshaded}\begin{oldverbatim}}{\end{oldverbatim}\end{myshaded}}
%
\newcommand{\KeywordTok}[1]{\hlkwd{#1}}
\newcommand{\DataTypeTok}[1]{\hlkwc{#1}}
\newcommand{\DecValTok}[1]{\hlnum{#1}}
\newcommand{\BaseNTok}[1]{\hlnum{#1}}
\newcommand{\FloatTok}[1]{\hlnum{#1}}
\newcommand{\ConstantTok}[1]{\hlnum{#1}}
\newcommand{\CharTok}[1]{\hlstr{#1}}
\newcommand{\SpecialCharTok}[1]{\hlstr{#1}}
\newcommand{\StringTok}[1]{\hlstr{#1}}
\newcommand{\VerbatimStringTok}[1]{\hlstr{#1}}
\newcommand{\SpecialStringTok}[1]{\hlstr{#1}}
\newcommand{\ImportTok}[1]{{#1}}
\newcommand{\CommentTok}[1]{\hlcom{#1}}
\newcommand{\DocumentationTok}[1]{\hlcom{#1}}
\newcommand{\AnnotationTok}[1]{\hlcom{#1}}
\newcommand{\CommentVarTok}[1]{\hlcom{#1}}
\newcommand{\OtherTok}[1]{{#1}}
\newcommand{\FunctionTok}[1]{\hlstd{#1}}
\newcommand{\VariableTok}[1]{\hlstd{#1}}
\newcommand{\ControlFlowTok}[1]{\hlkwd{#1}}
\newcommand{\OperatorTok}[1]{\hlopt{#1}}
\newcommand{\BuiltInTok}[1]{{#1}}
\newcommand{\ExtensionTok}[1]{{#1}}
\newcommand{\PreprocessorTok}[1]{\textit{#1}}
\newcommand{\AttributeTok}[1]{{#1}}
\newcommand{\RegionMarkerTok}[1]{{#1}}
\newcommand{\InformationTok}[1]{\textcolor{messagecolor}{#1}}
\newcommand{\WarningTok}[1]{\textcolor{warningcolor}{#1}}
\newcommand{\AlertTok}[1]{\textcolor{errorcolor}{#1}}
\newcommand{\ErrorTok}[1]{\textcolor{errorcolor}{#1}}
\newcommand{\NormalTok}[1]{\hlstd{#1}}
%
\AtBeginDocument{\bibliographystyle{/data/nasif12/modules_if12/SL7/i12g/R/4.0.1-Bioc3.11/lib64/R/library/BiocStyle/resources/tex/unsrturl}}


\begin{document}
\maketitle


{
\setcounter{tocdepth}{2}
\tableofcontents
\newpage
}
\hypertarget{vectors}{%
\section{Vectors}\label{vectors}}

First, create three named numeric vectors of size 10, 11 and 12 respectively in the following manner:

\begin{enumerate}
\def\labelenumi{\arabic{enumi})}
\tightlist
\item
  One vector with the ``colon'' approach: \emph{from:to}
\item
  One vector with the \texttt{seq()} function: \emph{seq(from, to)}
\item
  And one vector with the \texttt{seq()} function and the \texttt{by} argument: \emph{seq(from, to, by)}
\end{enumerate}

For easier naming you can use the vector \texttt{letters} or \texttt{LETTERS} which contain the latin alphabet in small and capital, respectively. In order to select specific letters just use e.g.~\texttt{letters{[}1:4{]}} to get the first four letters.
Check their types. What is the outcome? Where do you think the difference comes from?

\hypertarget{factors}{%
\section{Factors}\label{factors}}

\begin{enumerate}
\def\labelenumi{\arabic{enumi})}
\tightlist
\item
  Create a character vector consisting of three annotations \emph{Mutant-1, Mutant-2, Control}.
\item
  Using this annotation vector, create a factor where each annotation is repeated 4 times in a sequential manner (\emph{Mutant-1, Mutant-2, Control, Mutant-1, Mutant-2, Control, \ldots{}}). In addition, the levels are the sorted annotation values.
\item
  Print the results.
\end{enumerate}

\hypertarget{data-tables}{%
\section{Data tables}\label{data-tables}}

The purpose of this exercise is to get familiarized with data.table and try out some of its useful features.

\hypertarget{basic-operations}{%
\subsection{Basic operations}\label{basic-operations}}

Please follow the steps listed below:

\begin{enumerate}
\def\labelenumi{\arabic{enumi})}
\item
  load the library called \emph{dslabs}
\item
  Access the database called \emph{brexit\_polls}. You can take a look at the the \emph{help} documentation of this database (\emph{?brexit\_polls}) to learn about its content.
\end{enumerate}

For example:

\begin{longtable}[]{@{}ll@{}}
\toprule
column name & Description\tabularnewline
\midrule
\endhead
pollster & Pollster conducting the poll.\tabularnewline
poll\_type & Online or telephone poll.\tabularnewline
samplesize & Sample size of poll.\tabularnewline
remain & Proportion voting Remain.\tabularnewline
leave & Proportion voting Leave.\tabularnewline
\bottomrule
\end{longtable}

\begin{enumerate}
\def\labelenumi{\arabic{enumi})}
\setcounter{enumi}{2}
\item
  Inspect this data by checking properties such as the class type, the number of rows and columns, its column names, the unique values in the \emph{poll\_type} column.
\item
  Create a new variable called \emph{brexit\_DT} and assign the data.table converted version of \emph{brexit\_polls}.
\end{enumerate}

\hypertarget{more-exciting-operations}{%
\subsection{More exciting operations}\label{more-exciting-operations}}

Continue from the previous part and perform the following actions:

\begin{enumerate}
\def\labelenumi{\arabic{enumi})}
\setcounter{enumi}{4}
\item
  From \emph{brexit\_DT} get the counts of Online and Telephone polls
\item
  What are the mean and median values of the \emph{samplesize}
\item
  Add a new column \emph{remain\_polls} to \emph{brexit\_DT} that holds the multiplication of \emph{samplesize} to \emph{remain}
\item
  What is the range of values in this newly created column?
\item
  How do the mean values of \emph{undecided} look like when grouped by \emph{pollster}? How do they look like when grouped by \emph{poll\_type}? What is this mean value when \emph{pollster} is \emph{YouGov}?
\item
  Remove the column \emph{remain\_polls} created in step 7.
\end{enumerate}

\hypertarget{looping}{%
\section{Looping}\label{looping}}

\begin{enumerate}
\def\labelenumi{\arabic{enumi})}
\tightlist
\item
  Initialize a variable called \emph{counter} by 0.
\item
  Using a for loop that iterates 10 times,
\end{enumerate}

\begin{itemize}
\tightlist
\item
  create a random number drawn from a uniform distribution with \emph{min=0} and \emph{max=5}.
\item
  whenever this random number is bigger than or equal to 1, increment \emph{counter} by 1.
\end{itemize}

\begin{enumerate}
\def\labelenumi{\arabic{enumi})}
\setcounter{enumi}{2}
\tightlist
\item
  Print the final value in \emph{counter}.
\end{enumerate}

\hypertarget{functions}{%
\section{Functions}\label{functions}}

\begin{enumerate}
\def\labelenumi{\arabic{enumi})}
\tightlist
\item
  Write a function named \emph{get\_counts} that takes a GTEx data table as input and outputs the total counts of rows that the sample tissue type (\emph{SMTS}) is \emph{Heart} and the sample analysis freeze (\emph{SMAFRZE}) is \emph{RNASEQ}.
\item
  How about if you try the same but for \emph{Blood}.
\end{enumerate}

\begin{itemize}
\tightlist
\item
  If this task was too easy, can you modify your function such that instead of taking only one argument, it takes two additional ones, one for the \emph{SMTS} and another for \emph{SMAFRZE}. Iterate over all possible values of \emph{SMTS} (\textbf{Hint:} \emph{unique(data\$SMTS)}) and call your function by providing the sample tissue type.
\end{itemize}

\hypertarget{r-markdown}{%
\section{R Markdown}\label{r-markdown}}

Downloaded and stored the \emph{sample\_annotation.tsv} file from Google drive. Then, create an Rmarkdown file and perform the following tasks:
1) Read the \emph{sample\_annotation.tsv} file.
2) Create a new variable containing the counts of each \emph{tissue} existing in the data.
3) Use the \emph{barplot} function to plot the number of tissue types in the GTEx data.
4) Try to sort the bars according to the tissue counts (optional).


\end{document}
